\documentclass{article}

\usepackage[utf8]{inputenc}

\begin{document}

\begin{titlepage}
\author{Gennadi Heimann} 
\title{Change Log} 
\date{29.12.2017} 
\maketitle
\end{titlepage}

\section{v1.0.0}

\noindent \textbf{@created on 09.11.2017}\\
\textbf{@finished on}\\

\noindent Bei der Erstellung von einem Schritt muss sichergestellt werden, dass nur einen 
Schritt zu der Komponente verbunden wird\\

\noindent \textbf{@created on 24.11.2017}\\
\textbf{@finished on }\\

\noindent Edge von Config zu FirstStep muss hasStep heißen\\

\noindent \textbf{@created on 09.11.2017}\\
\textbf{@finished on}\\

\noindent In SelectionCriterium darf nicht min > max sein\\

\noindent \textbf{@created on 30.01.2018}\\
\textbf{@finished on }\\

\noindent Globale Einstellungen fuer die Konfiguration.
Der Vertex Config bekommt einen zusaetzlichen Vertex <GlobalSettings> angehaengt.
In <GlobalSettings> werden die Einstellungen fuer gesamte Konfiguration definiert.
Folgende Einstellungen werden gemacht configurationCourse: [sequence or sabstitute] :
\begin{itemize}
	\item sequence\\ 
	Die Schritte werden hintereinander ausgeführt. Die abgearbeitete Schritte 
	werden nicht aus der GUI entfernt. Der Benutzer kann jederzeit aus der schon bearbeteten
	Schritt die Komponente entfernen oder bearbeiten.\\
   \item sabstitute\\ 
   Die Schritte werden hintereinader ausgeführt. Die abgearbeitete Schritt 
	wird aus der GUI entfernt und danach neue geladen. Das Zurückkehern wird über der Aktion
   des Benutzers ausgeführt. Der Benutzer sieht nur einen aktuellen Schritt in der GUI.\\
\end{itemize}

\noindent \textbf{@created on 10.04.2018}\\
\textbf{@finished on }\\

\noindent Beim nicht erreichtem Server Error an Client senden. Connection Error fangen.\\

\noindent \textbf{@created on 18.04.2018}\\
\textbf{@finished on }\\

\noindent Status un BO von Registration und Login zu dem UserBO und StatusUser zusammenfuehren\\

\section{v1.0.1}

\noindent \textbf{@created on 15.05.2017}\\
\textbf{@finished on }\\

\noindent Implementierung des Login\\
Docu zu der Login.\\
https://www.lightbend.com/activator/template/bootzooka\\
http://www.lightbend.com/activator/template/play-oauth2-scala\\
https://scalaplayschool.wordpress.com/2014/08/27/lesson-12-a-login-action-with-scala-play-angular-bootstrap/\\
https://www.playframework.com/documentation/2.0.4/ScalaSecurity\\
https://github.com/nezasa/play-login-sample\\
https://www.playframework.com/documentation/2.1.1/JavaGuide4\\
http://ics-software-engineering.github.io/play-example-login/\\
        
\end{document}

























